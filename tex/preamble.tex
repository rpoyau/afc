% ===================== AFC Mark 2 — preamble (pdfLaTeX) ======================

% Core math and layout
\usepackage{amsmath,amssymb,amsthm,mathtools,bm}
\usepackage{enumitem}
\usepackage{microtype}
\usepackage[margin=1in]{geometry}
\usepackage{hyperref}

% ================= Unicode + code safety (pdfLaTeX) =================
% Keep sources unchanged; map the few Unicode glyphs used in the project.

\usepackage[utf8]{inputenc}
\usepackage[T1]{fontenc}
\usepackage{textcomp}       % extra text symbols (e.g., en/em dash handling)
\usepackage{newunicodechar}

% Greek letters actually appearing in the sources
\newunicodechar{Ω}{\ensuremath{\Omega}}
\newunicodechar{α}{\ensuremath{\alpha}}
\newunicodechar{Δ}{\ensuremath{\Delta}}

% Arrows and relations
\newunicodechar{⇒}{\Rightarrow}
\newunicodechar{→}{\ensuremath{\rightarrow}}
\newunicodechar{↔}{\ensuremath{\leftrightarrow}}

% Comparators and math symbols
\newunicodechar{≥}{\ensuremath{\ge}}
\newunicodechar{≤}{\ensuremath{\le}}
\newunicodechar{±}{\ensuremath{\pm}}
\newunicodechar{×}{\ensuremath{\times}}

% Typography
\newunicodechar{—}{\textemdash}   % em dash
\newunicodechar{–}{\textendash}   % en dash
\newunicodechar{…}{\ldots}        % ellipsis

% Project convenience macros
\newcommand{\AO}{A\ensuremath{\Omega}} % use \AO{} instead of raw “AΩ” if you like

% Code-like tokens with underscores/braces, safe even in headings/captions:
\newcommand{\code}[1]{\texttt{\detokenize{#1}}}
% Example: \code{R_3}, \code{duon_{2n}}, \code{dark-ual_{2n}}
% ====================================================================

% --------------------- Theorem-like environments ----------------------------
\theoremstyle{definition}
\newtheorem{axiom}{Axiom}
\newtheorem{definition}{Definition}
\newtheorem{choice}{Choice}
\newtheorem{consequence}{Consequence}
\newtheorem{theorem}{Theorem}
\newtheorem{lemma}{Lemma}
\newtheorem{corollary}{Corollary}
\newtheorem{remark}{Remark}
\newtheorem{falsifier}{Falsifier}

% --------------------- Project convenience symbols --------------------------
\newcommand{\bip}{\mathrm{bip}}
\newcommand{\Z}{\mathbb{Z}}
\newcommand{\trits}{\{-1,0,+1\}}

% ============================================================================
