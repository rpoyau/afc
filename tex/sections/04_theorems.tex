\section{Corollaries and Theorems}\label{sec:theorems}

\subsection{TXOR Update and Admissibility}

\begin{theorem}[TXOR Update]\label{thm:txor}
For each node $x$ and bip $t \to t+1$, the ledger update can be written in TXOR form as
\[
    n_{t+1}(x)
    = n_t(x) \oplus_3 \tau_{\mathrm{in}}(x) \oplus_3 \bigl(-\tau_{\mathrm{out}}(x)\bigr),
\]
where $\oplus_3$ is TXOR from Definition~\ref{def:txor}, and
\[
    \tau_{\mathrm{in}}(x) := \sum_{e:\, e \to x} \tau(e), 
    \qquad
    \tau_{\mathrm{out}}(x) := \sum_{e:\, x \to e} \tau(e).
\]
\end{theorem}

\begin{proof}[Sketch]
By Axiom~\ref{ax:ternary}, node ledgers take values in $\mathbb{Z}_3$ and outcomes $\tau(e)$ lie in $T = \{-1,0,+1\}$. By Axiom~\ref{ax:seats}, each seat contributes at most one $\tau(e)$ per bip. By Axiom~\ref{ax:update}, for each node $x$,
\[
    n_{t+1}(x) = \bigl( n_t(x) + \tau_{\mathrm{in}}(x) - \tau_{\mathrm{out}}(x) \bigr) \bmod 3.
\]
Using Definition~\ref{def:txor}, addition modulo $3$ is TXOR, so the stated form follows.
\end{proof}

\begin{consequence}[Admissibility Logic]\label{cons:admissibility}
The node update at $x$ is non-trivial in a step $t \to t+1$ only if there is at least one seat incident on $x$ with non-zero flux. By Definition~\ref{def:flux} and Axiom~\ref{ax:stochastic}, non-zero flux on a seat requires both a non-zero stochastic feed $S$ and an open gate $g=1$. The admissibility of motion at $x$ is therefore determined by the gated outcomes $\tau$ and the TXOR update in Theorem~\ref{thm:txor}.
\end{consequence}

\subsection{Counts-First Stokes and Global Conservation}

\begin{theorem}[Counts-First Stokes]\label{thm:stokes}
Let $U$ be a finite set of nodes with boundary chain $\partial U$ consisting of seats with exactly one endpoint in $U$. For each boundary seat $e \in \partial U$, let $\tau_{\mathrm{in}}(e)$ denote the flux counted inward into $U$. Then
\[
    \sum_{e \in \partial U} \tau_{\mathrm{in}}(e)
    = \sum_{x \in U} \Delta n(x),
\]
where $\Delta n(x)$ is the production from Definition~\ref{def:production}.
\end{theorem}

\begin{proof}[Sketch]
By Definition~\ref{def:production} and Axiom~\ref{ax:update},
\[
    \Delta n(x) = \sum_{e:\, e \to x} \tau(e) - \sum_{e:\, x \to e} \tau(e).
\]
Summing over all $x \in U$, every internal seat with both endpoints in $U$ appears once with $+\tau(e)$ and once with $-\tau(e)$, so its contributions cancel pairwise. Seats with exactly one endpoint in $U$ contribute a single term, with sign fixed by the inward orientation, yielding the stated sum over $\partial U$.
\end{proof}

\begin{consequence}[Global Stokes]\label{cons:global-stokes}
If a region $U$ has empty boundary ($\partial U = \emptyset$), then
\[
    \sum_{x \in U} \Delta n(x) = 0.
\]
In particular, for $U$ equal to the entire finite manifold, the total integer ledger is conserved exactly.
\end{consequence}

\subsection{Graph Hodge Decomposition}

\begin{consequence}[Potential + Harmonic Split]\label{cons:hodge}
Let $J$ denote the expected flux field on seats of a connected finite graph and let $\delta J$ denote its discrete divergence on nodes. There exists an integer node potential $\Phi$ and a flux field $h$ such that
\[
    J = \nabla \Phi + h,
\]
where $h$ is divergence-free ($\delta h = 0$) and supported on the cycle space of the graph. The decomposition is unique up to an additive constant in $\Phi$. On simply connected patches (no non-trivial cycles), the harmonic component vanishes ($h=0$).
\end{consequence}

\subsection{Run-and-Tumble Limit}

\begin{choice}[Memorum $M_{14}$]\label{choice:memorum}
For a declared finite sequence of seats (a diatia chain), define the Memorum as the cumulative ternary bias
\[
    M_{14} := \sum_{k=1}^{L} \tau_k,
\]
where $\tau_k \in T$ is the realized flux on the $k$-th seat of the chain in a given history. The integer $M_{14}$ is used to bias the kernel $S$ of Axiom~\ref{ax:stochastic} on subsequent traversals of the same chain.
\end{choice}

\begin{consequence}[Run-and-Tumble Limit]\label{cons:rtp}
Consider a family of diatia chains with Memorum bias as in Choice~\ref{choice:memorum}. On a macroscopic scale ($N \gg 1$ bips) in a diffusion-type scaling limit, the path ensemble exhibits:
\begin{itemize}
    \item a drift component arising from the biased kernel induced by $M_{14}$ (run), and
    \item a diffusion component arising from the symmetric part of the stochastic pushforward (tumble) from Axiom~\ref{ax:stochastic}.
\end{itemize}
In this limit, the macroscopic path distribution takes a Feynman--Kac--type form, built from counts of discrete trajectories on the finite manifold.
\end{consequence}
