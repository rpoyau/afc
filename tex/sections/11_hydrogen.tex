\section{Example: Hydrogen-Like Radial Shells}\label{sec:hydrogen-example}

\begin{definition}[Radial shell plan]\label{def:radial-plan}
A radial shell plan is a finite plan $(V,E)$ together with:
\begin{itemize}
    \item a set of radii $R := \{1,2,\dots,R_{\max}\}$;
    \item a partition $V = \bigsqcup_{r \in R} L_r$, where $L_r$ is the set of nodes at radius index $r$;
    \item seats $E$ that connect nodes within each $L_r$ and between neighbouring shells $L_r$ and $L_{r\pm 1}$.
\end{itemize}
A central source is represented by a distinguished region associated with radius index $0$ and is not used explicitly in the update rules.
\end{definition}

\begin{definition}[Electron body and internal state]\label{def:electron-body}
On a radial shell plan as in Definition~\ref{def:radial-plan}, a single electron is represented by a body node $b \in V$ with internal state
\[
    s_t(b) = (r_t,\phi_t),
\]
where $r_t \in R$ is a radial index and $\phi_t \in \mathbb{R}/(2\pi\mathbb{Z})$ is a phase register at bip $t$. The full $n$-body state $(V,E,B,n_t,s_t)$ is as in Definition~\ref{def:nbody-state}, with $B$ containing $b$ and any additional bodies that represent the environment.
\end{definition}

\begin{definition}[Radial phase cost and resonance score]\label{def:radial-cost-resonance}
A radial phase cost is a map
\[
    c : R \to \mathbb{R}_{>0}, \qquad r \mapsto c(r),
\]
interpreted as the phase increment per bip at radius $r$. Given a phase register $\phi \in \mathbb{R}/(2\pi\mathbb{Z})$, the resonance score is
\[
    S(\phi) := 1 - \frac{2}{\pi}\,\min\{\phi,\,2\pi - \phi\},
\]
so that $S(\phi) = 1$ at $\phi \equiv 0 \pmod{2\pi}$ and $S(\phi) = 0$ at $\phi \equiv \pi \pmod{2\pi}$.
\end{definition}

\begin{consequence}[Specialisation of the $n$-body update]\label{cons:hydrogen-nbody}
On the radial shell plan of Definition~\ref{def:radial-plan} with electron body $b$ and internal state $(r_t,\phi_t)$ as in Definition~\ref{def:electron-body}, the $n$-body update (Consequence~\ref{cons:nbody-step}) is specialised as follows for $b$:
\begin{enumerate}
    \item Outcomes $\tau_t(e)$ are sampled on all seats $e \in E$ using declared kernels. In particular, outcomes on seats connecting $L_{r_t}$ to $L_{r_t\pm 1}$ determine a proposed radial increment $\delta r_t \in \{-1,0,+1\}$ via a local map that depends only on $\{\tau_t(e): e \text{ incident to } b\}$.
    \item The ledger $n_t$ is updated to $n_{t+1}$ at all nodes by
    \[
        n_{t+1}(x) = n_t(x) \oplus_3 \Delta n_t(x).
    \]
    \item The internal state of $b$ is updated by a local map
    \[
        s_{t+1}(b)
        =
        F_b\bigl(
            b,\,
            s_t(b),\,
            n_t(b),\,
            \Delta n_t(b),\,
            \{\tau_t(e): e \text{ incident to } b\}
        \bigr),
    \]
    where $F_b$ acts on the components by
    \begin{align*}
        r'_{t} &:= \min\{\max\{r_t + \delta r_t, 1\}, R_{\max}\}, \\
        \phi'_{t} &:= \phi_t + c(r'_{t}) \pmod{2\pi}, \\
        s_{t+1}(b) &:= 
        \begin{cases}
            (r'_{t}, \phi'_{t}) & \text{with probability } S(\phi'_{t}),\\
            (r_t, \phi_t)       & \text{with probability } 1 - S(\phi'_{t}).
        \end{cases}
    \end{align*}
\end{enumerate}
The path $(r_t,\phi_t)_{t\ge 0}$ of the electron body is generated by the local $n$-body update with this internal state map $F_b$.
\end{consequence}

\begin{definition}[Radial occupation]\label{def:radial-occupation}
For a time window of $T$ bips, the radial occupation counts are
\[
    N_r(T) := \bigl|\{\, t \in \{0,\dots,T-1\} : r_t = r \,\}\bigr|
    \quad\text{for } r \in R.
\]
These counts define a histogram over the radial shell indices.
\end{definition}

\begin{remark}[Shell structure]\label{rem:hydrogen-shells}
For a fixed phase cost profile $c(r)$ and internal state map $F_b$ as in Consequence~\ref{cons:hydrogen-nbody}, radial occupation histograms $N_r(T)$ exhibit peaks at radii where the accumulated phase admits near-integer loop closure at the level of the local acceptance rule. These peaks appear as preferred radial shells in the recorded path of the electron body.
\end{remark}
