\section{Example: Two-Body Plan (Sun--Mercury)}\label{sec:sun-mercury}

\begin{definition}[Two-body ring plan]\label{def:two-body-plan}
A two-body ring plan is a finite plan $(V,E)$ together with:
\begin{itemize}
    \item a cycle $C_N$ with node set $\{0,1,\dots,N-1\}$ and oriented seats
    \[
        e_i : i \to (i+1) \bmod N, \quad i = 0,\dots,N-1;
    \]
    \item a distinguished central node $\sigma$ (Sun);
    \item a body set $B = \{\sigma,M\}$, where $M$ is the Mercury body;
    \item a node ledger $n_t : V \to \mathbb{Z}_3$ and an internal state map
    \[
        s_t : B \to S,
    \]
    where $S$ contains a phase index component $k_t \in \{0,\dots,N-1\}$ for $M$, indicating its position on $C_N$ at bip $t$.
\end{itemize}
The global evolution of $(V,E,B,n_t,s_t)$ per bip is given by the $n$-body update in Consequence~\ref{cons:nbody-step}.
\end{definition}

\begin{definition}[Orbital kernel and phase increment]\label{def:mercury-kernel}
Let $T=\{-1,0,+1\}$ be the ternary alphabet. For the Mercury body $M$, declare:
\begin{itemize}
    \item a ternary kernel $K_{\mathrm{orb}}$ on $T$ used to sample outcomes on tangential seats incident to $M$;
    \item a map
    \[
        G_{\mathrm{orb}} : T^{d_M} \to \{-1,0,+1\},
    \]
    where $d_M$ is the number of tangential seats incident to $M$, mapping the incident outcomes at bip $t$ to a phase increment $\Delta k_t$.
\end{itemize}
The expected phase increment per bip at the chosen ring radius is set by calibration to a target substrate orbital frequency $\nu_{\mathrm{sub}}$ as in Section~\ref{sec:units}, so that
\[
    \mathbb{E}[\Delta k_t] = N\,\nu_{\mathrm{sub}} \bmod N.
\]
\end{definition}

\begin{consequence}[Specialisation of the $n$-body update]\label{cons:sun-mercury-nbody}
On the two-body ring plan of Definition~\ref{def:two-body-plan}, the $n$-body update (Consequence~\ref{cons:nbody-step}) is specialised as follows for $M$:
\begin{enumerate}
    \item Outcomes $\tau_t(e)$ are sampled on all seats $e \in E$ using the kernels declared on the plan, including $K_{\mathrm{orb}}$ on tangential seats incident to $M$.
    \item The ledger $n_t$ is updated to $n_{t+1}$ at all nodes by
    \[
        n_{t+1}(x) = n_t(x) \oplus_3 \Delta n_t(x).
    \]
    \item The internal state of $M$ is updated by a local map
    \[
        s_{t+1}(M)
        =
        F_M\bigl(
            M,\,
            s_t(M),\,
            n_t(M),\,
            \Delta n_t(M),\,
            \{\tau_t(e): e \text{ incident to } M\}
        \bigr),
    \]
    where $F_M$ acts on the phase component by
    \[
        k_{t+1} := (k_t + \Delta k_t) \bmod N,
    \]
    with $\Delta k_t = G_{\mathrm{orb}}(\{\tau_t(e)\}_{e \text{ tangential to } M})$.
\end{enumerate}
The Sun body $\sigma$ may have a static internal state or a declared internal evolution; this does not change the form of the update.
\end{consequence}

\begin{definition}[Orbital measurement]\label{def:mercury-measurement}
For a time window of $T$ bips, the phase history of $M$ is the sequence $(k_t)_{t=0}^{T-1}$ extracted from $s_t(M)$. The empirical substrate orbital frequency is
\[
    \nu_{\mathrm{meas}}(T)
    :=
    \frac{1}{2\pi T}
    \sum_{t=0}^{T-1}
        \bigl(\theta_{t+1} - \theta_t\bigr) \bmod 2\pi,
\]
where $\theta_t := 2\pi k_t/N$. Using the calibration in Section~\ref{sec:units}, this can be reported as a physical orbital period for comparison with external Sun--Mercury data.
\end{definition}

\begin{remark}[Reference implementation]\label{rem:sun-mercury-impl}
An external script can realise Consequence~\ref{cons:sun-mercury-nbody} by:
\begin{itemize}
    \item sampling $\tau_t(e)$ on the ring plan using the declared kernels;
    \item updating $n_{t+1}$ and $s_{t+1}$ via the generic $n$-body step;
    \item reading out $(k_t)$ from $s_t(M)$ and mapping to angles $\theta_t$ and planar coordinates for plotting.
\end{itemize}
The two-body Sun--Mercury test is thus implemented as a special case of the general $n$-body calculator.
\end{remark}
