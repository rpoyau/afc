% sections/01_abstract.tex

\begin{center}
  \textit{Axioms $\to$ Distinctions $\to$ Relations $\to$ Motifs $\to$ Math.}\\[0.25em]
  \textit{This is the way.}
\end{center}



\begin{abstract}
A relational form of the discrete Stokes identity is constructed within the
Axiomatic Fundamentalism (AF) protocol~\cite{reginald2025af}. The starting
point is a single relational axiom: a Null state $\mathcal{N}$, denoted
$\mathcal{N} \equiv \infty \equiv \chi$, carries no distinctions and no
relations in the model universe.

Subsequent structure is expressed in terms of finite families of distinctions.
These families support a stochastic process, represented by a Markov kernel
$P$ on the space of distinctions, together with associated flux quantities
$\phi_{ij}$ and divergences $\mathrm{div}(i)$ defined as bookkeeping on the
induced transitions. For any distinguished Region $R$ (an inside/outside cut
formed by grouping distinctions), the construction yields an identity of the
form
\[
  \sum_{i \in R} \mathrm{div}(i)
  =
  \sum_{(i,j) \in \partial R} \phi_{ij},
\]
the AFC Stokes identity, which expresses a continuity relation across the
chosen relational cut.

Under a dictionary between distinctions, flux, and divergence and the usual
chains, cochains, and coboundaries, this identity acquires the algebraic form
of the classical discrete generalized Stokes theorem on cell complexes as
treated in standard sources on geometric integration and discrete exterior
calculus. Information-theoretic capacity bounds for the induced Markov
structure are also obtained; these bounds are derived within the same
relational framework and are not assumed as additional axioms.
\end{abstract}
