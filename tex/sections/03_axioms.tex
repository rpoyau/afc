% sections/02_axioms.tex

\section{Axiom: Null = Infinity}
\label{sec:axiom}

\begin{axiom}[Null = Infinity]
\label{ax:null-infinity}
There exists a distinguished Null state $\mathcal{N}$ which carries no
distinctions and no relations in the model universe. Within this document,
the symbols $\infty$ and $\chi$ are used synonymously with $\mathcal{N}$:
\[
  \mathcal{N} \equiv \infty \equiv \chi.
\]
\end{axiom}

\begin{consequence}[Distinctions and Regions]
\label{cons:region}
Any use of a ``Region'' $R$ uses at least one distinction that separates an
``inside'' collection of positions from a complementary ``outside'' (the Null
background). A Region $R$ is defined as a \textbf{finite} family of positions
marked against the Null baseline.
\begin{remark}
This structure corresponds to the AF closure condition RF2 in
Reginald~\cite{reginald2025af}: within the model, nothing is taken to exist
beyond the declared Distinctions and their groupings.
\end{remark}
\end{consequence}

\begin{consequence}[Relations from Distinctions]
\label{cons:relations}
Relations have no independent existence. Once Distinctions are present,
Relations arise only as connections (directed or undirected) between the
distinguished positions.
\begin{remark}
This corresponds to RF1 in Reginald~\cite{reginald2025af}: Distinction
precedes relation, and no ``absolute'' relations are assumed prior to
Distinctions.
\end{remark}
\end{consequence}
