\section{Navier--Stokes Usage Example}\label{sec:navier-stokes}

\begin{definition}[Stokes mesh]\label{def:stokes-mesh}
A Stokes mesh on a finite directed graph $(V,E)$ consists of:
\begin{itemize}
    \item a distinguished subset $U \subseteq V$ of nodes (the fluid region);
    \item a distinguished subset $V_{\partial} \subseteq V$ of boundary nodes;
    \item for each finite region $W \subseteq U$, an oriented boundary chain $\partial W \subseteq E$;
\end{itemize}
such that for every expected current $J : E \to \mathbb{R}$ and production $\sigma : V \to \mathbb{R}$ induced from AFC outcomes, the counts-first Stokes identity
\[
    \sum_{e \in \partial W} J(e)
    =
    \sum_{x \in W} \sigma(x)
\]
holds for all regions $W \subseteq U$.
\end{definition}

\begin{definition}[Fluid plan]\label{def:fluid-plan}
A \emph{fluid plan} is a finite directed graph $(V,E)$ together with:
\begin{itemize}
    \item a partition of $V$ into fluid nodes $V_{\mathrm{fl}}$ and boundary nodes $V_{\partial}$, with $V_{\mathrm{fl}} \subseteq U$ in a Stokes mesh as in Definition~\ref{def:stokes-mesh};
    \item a Stokes mesh structure representing a fluid domain $U$ and its boundary chains $\partial W$ for regions $W \subseteq U$.
\end{itemize}
Each fluid node $x \in V_{\mathrm{fl}}$ represents a control volume $C_x \subseteq U$, and each seat $e = (u \to v) \in E$ represents an oriented interface between neighbouring volumes or between a volume and the boundary.
\end{definition}

\begin{definition}[Discrete density, current, and velocity]\label{def:fluid-fields}
On a fluid plan, a \emph{mass density} is a map
\[
    \rho : V_{\mathrm{fl}} \to \mathbb{R}_{\ge 0},
\]
and an \emph{expected current} is a map
\[
    J : E \to \mathbb{R}^d,
\]
for $d \in \{2,3\}$, obtained as a windowed average of ternary fluxes as in Definition~\ref{def:expected-current}. The discrete divergence $\delta J$ is taken componentwise. A \emph{velocity} field at nodes is defined by
\[
    u(x) := \frac{1}{\rho(x)} \sum_{e: x \to y} J(e)
\]
for $\rho(x) > 0$ and $x \in V_{\mathrm{fl}}$.
\end{definition}

\begin{definition}[Graph Laplacian on nodes]\label{def:laplacian}
On a fluid plan, the \emph{graph Laplacian} acting on node fields $f : V_{\mathrm{fl}} \to \mathbb{R}^d$ is
\[
    (\Delta f)(x)
    :=
    \sum_{y \sim x} w_{xy}\,\bigl(f(y) - f(x)\bigr),
\]
where the sum runs over neighbours $y$ of $x$ and $w_{xy} \ge 0$ are symmetric weights encoding the geometry of the mesh and any declared viscosity profile.
\end{definition}

\begin{consequence}[Discrete Navier--Stokes step]\label{cons:discrete-ns}
On a fluid plan with density $\rho$, current $J$, and velocity $u$ as in Definition~\ref{def:fluid-fields}, a single bip update of an incompressible flow is represented by:
\begin{enumerate}
    \item \textbf{Advection.} A provisional velocity $u^\ast$ is obtained from $u$ and $J$ by a local rule using only values of $u$ at neighbours of each node $x \in V_{\mathrm{fl}}$ and the current across seats incident to $x$.
    \item \textbf{Diffusion.} A diffused velocity $u^{\ast\ast}$ is obtained by
    \[
        u^{\ast\ast}(x) = u^\ast(x) + \nu\,\Delta t\,(\Delta u)(x),
    \]
    with $\Delta$ the graph Laplacian of Definition~\ref{def:laplacian} and kinematic viscosity parameter $\nu \ge 0$.
    \item \textbf{Projection.} The current $J$ is adjusted so that
    \[
        \delta J = 0 \quad \text{on } V_{\mathrm{fl}},
    \]
    as in Consequence~\ref{cons:div-current-eq}, and the node velocity is updated from this divergence-free current. The projection is computed using only discrete divergences and boundary sums on the Stokes mesh.
\end{enumerate}
Each substep is expressed in terms of AFC currents, divergences, and the Stokes structure of the fluid plan.
\end{consequence}

\begin{theorem}[Stokes-consistent update]\label{thm:ns-consistent}
Let a fluid plan and discrete fields $(\rho,J,u)$ be given as above. Suppose the update of Consequence~\ref{cons:discrete-ns} is implemented by AFC rules that preserve the counts-first divergence equation $\delta J = 0$ on $V_{\mathrm{fl}}$ at each bip. Then, for every finite region $W \subseteq V_{\mathrm{fl}}$ with boundary chain $\partial W$ in the Stokes mesh,
\[
    \sum_{e \in \partial W} J(e) = 0
\]
holds after every bip. Under a discrete-to-GM mapping and refinement of the fluid plan and bip duration, the evolution of the node velocity agrees with an incompressible Navier--Stokes balance on the associated continuum domain while preserving the GM Stokes identity.
\end{theorem}

\begin{remark}[Algorithmic and parallel scaling]\label{rem:ns-scaling}
Let $N := |V_{\mathrm{fl}}|$ be the number of fluid nodes and assume $|E| = O(N)$ seats with a uniformly bounded number of neighbours per node. In Consequence~\ref{cons:discrete-ns}:
\begin{itemize}
    \item the advection and diffusion substeps are local and can be evaluated independently at each node from its neighbours, with total work $O(N)$ and parallel depth $O(1)$ per bip on an ideal parallel machine;
    \item the projection substep uses discrete divergences and boundary sums on the Stokes mesh and can be implemented by hierarchical or multilevel methods with total work $O(N \log N)$ and logarithmic parallel depth in $N$.
\end{itemize}
The overall cost per bip is $O(N)$ in a sequential implementation and can be organised as $O(N \log N)$ work with polylogarithmic depth in a parallel implementation. These statements concern algorithmic scaling of the AFC update.
\end{remark}

\begin{remark}[Scope of the construction]\label{rem:ns-scope}
The structures and updates in this section specify a discrete AFC scheme for incompressible flow on a finite fluid plan. Existence and uniqueness of the discrete evolution follow from the AFC update rules on a finite graph. Questions of existence, uniqueness, or regularity for the continuum Navier--Stokes system on unbounded or smooth domains lie outside the AFC axiom set and are not addressed here.
\end{remark}
