% corollaries/00_stokes.tex

\section{Corollary: Null Potential, Observer Coupling, and Discrete Stokes}
\label{sec:null-stokes-corollary}

\begin{corollary}[AFC Stokes Identity]
\label{cor:afc-stokes}
Let $V$ be a finite family of Distinctions with Memory State $p$, Markov
Kernel $P$, net flux $\phi$, and divergence $\mathrm{div}$ as in
Choices~\ref{choice:memory-state}--\ref{choice:divergence}. For any Region
$R \subseteq V$,
\begin{equation}
\label{eq:afc-stokes-cor}
  \sum_{i\in R} \mathrm{div}(i)
  =
  \sum_{\substack{i\in R \\ j\notin R}} \phi_{ij}.
\end{equation}
\end{corollary}

\noindent
\textbf{Derivation.}
From Consequences~\ref{cons:region}--\ref{cons:relations} and
Choices~\ref{choice:memory-state}--\ref{choice:region} by finite algebra on
a finite index set; no external postulates, in accordance with the AF
protocol of Reginald~\cite{reginald2025af}.

\begin{proof}
By Choice~\ref{choice:divergence},
\[
  \sum_{i\in R} \mathrm{div}(i)
  =
  \sum_{i\in R} \sum_{j\in V} \phi_{ij}
  =
  \sum_{i\in R} \sum_{j\in R} \phi_{ij}
  +
  \sum_{i\in R} \sum_{j\notin R} \phi_{ij}.
\]
For the internal term, use antisymmetry $\phi_{ij} = -\phi_{ji}$:
\[
  \sum_{i\in R} \sum_{j\in R} \phi_{ij}
  =
  \sum_{\{i,j\}\subset R} \bigl( \phi_{ij} + \phi_{ji} \bigr)
  = 0.
\]
Thus internal exchanges cancel. The remaining term is
\[
  \sum_{i\in R} \sum_{j\notin R} \phi_{ij}
  =
  \sum_{\substack{i\in R \\ j\notin R}} \phi_{ij},
\]
which is the net flux across the boundary of $R$ with sign convention taken
from inside to outside. This yields~\eqref{eq:afc-stokes-cor}.
\end{proof}

\subsection{Falsification Hooks}
\label{subsec:falsification}

For a Region $R \subseteq V$ define
\[
  S_{\mathrm{int}}(R) := \sum_{i\in R} \sum_{j\in R} \phi_{ij},
  \qquad
  S_{\mathrm{div}}(R) := \sum_{i\in R} \mathrm{div}(i),
  \qquad
  S_{\partial}(R) := \sum_{\substack{i\in R \\ j\notin R}} \phi_{ij}.
\]

\begin{itemize}
  \item \textbf{Internal cancellation.} Under
    Choice~\ref{choice:flux}, $\phi_{ij} = -\phi_{ji}$ implies
    $S_{\mathrm{int}}(R) = 0$ for all $R$. A nonzero value falsifies
    antisymmetry or the completeness of $R$.
  \item \textbf{Boundary balance.} The identity
    $S_{\mathrm{div}}(R) = S_{\partial}(R)$ is equivalent to
    \eqref{eq:afc-stokes-cor}. A persistent mismatch indicates that the
    Memory--Kernel pair $(p,P)$ or the family of Distinctions $V$ does not
    satisfy the Choices used in this construction.
\end{itemize}

\subsection{Mapping to Discrete Generalized Stokes}
\label{subsec:discrete-stokes-map}

On a finite cell complex, the discrete generalized Stokes theorem can be
written as
\begin{equation}
\label{eq:discrete-stokes}
  \sum_{\sigma\in\partial C} \omega(\sigma)
  =
  \sum_{\tau\in C} (d\omega)(\tau),
\end{equation}
where $C$ is a chain of cells, $\partial C$ its boundary, $\omega$ is a
cochain, and $d\omega$ its coboundary. This is the standard discrete Stokes
identity on cell complexes as treated, for example, in Whitney's geometric
integration theory, Munkres' algebraic topology, and the Discrete Exterior
Calculus literature~\cite{whitney1957,munkres1984,hirani2003,desbrun2005dec}.

Under the dictionary
\begin{center}
\begin{tabular}{ll}
  Region $R$                   & $\leftrightarrow$ chain $C$, \\
  net flux $\phi$ on boundary & $\leftrightarrow$ cochain $\omega$ on $\partial C$, \\
  divergence $\mathrm{div}$   & $\leftrightarrow$ coboundary $d\omega$ on $C$,
\end{tabular}
\end{center}
the AFC identity~\eqref{eq:afc-stokes-cor} has the same algebraic form as
\eqref{eq:discrete-stokes}. Classical proofs of~\eqref{eq:discrete-stokes}
(e.g.\ via incidence matrices or cochain algebra in the cited sources) then
apply to the mapped version, without entering the derivation of
Corollary~\ref{cor:afc-stokes}.
