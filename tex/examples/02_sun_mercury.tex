\section{Example: Two-Body Plan (Sun--Mercury)}\label{sec:sun-mercury}

\begin{definition}[Ring plan]\label{def:ring-plan}
Let $N \in \mathbb{N}$ with $N \ge 3$. A ring plan for a two-body configuration consists of:
\begin{itemize}
    \item a cycle graph $C_N$ with node set $\{0,1,\dots,N-1\}$ and oriented seats
    \[
        e_i : i \to (i+1) \bmod N, \quad i = 0,\dots,N-1;
    \]
    \item a phase index $k_t \in \{0,\dots,N-1\}$ at bip $t$, representing the angular position of the orbiting body on the ring;
    \item a distinguished central source represented by a fixed reference position (for example, the centre of the ring), not part of the cycle $C_N$.
\end{itemize}
The pair $(C_N,k_t)$ is the discrete phase space for the two-body example.
\end{definition}

\begin{remark}[Provenance of the ring]\label{rem:ring-provenance}
The ring plan $(C_N,E_{\mathrm{ring}})$ can be interpreted as the topological projection of a stable soliton cage induced by the central source. In such an interpretation, the ring length $N$ and the chosen substrate orbital frequency $\nu_{\mathrm{sub}}$ are determined by a refractive profile at the orbital radius, specified by an external refraction model. This provenance constrains the choice of $N$ and $\nu_{\mathrm{sub}}$ when matching a particular physical system; it is not required for the AFC phase update rule itself.
\end{remark}

\begin{definition}[Phase index and Memorum]\label{def:phase-memorum}
On the ring plan of Definition~\ref{def:ring-plan}, the phase index $k_t$ evolves over bips $t \in \mathbb{Z}$. For a time window $t = 0,\dots,T-1$, the associated Memorum for the orbit is
\[
    M_T := \sum_{t=0}^{T-1} \tau_t(e_{k_t}),
\]
where $\tau_t(e_{k_t}) \in T$ is the ternary outcome on the seat $e_{k_t}$ incident to the current phase index at bip $t$. The Memorum $M_T$ records the signed sequence of local outcomes along the realised orbital path.
\end{definition}

\begin{definition}[Ternary kernel with orbital bias]\label{def:orbital-kernel}
Let $T = \{-1,0,+1\}$ be the ternary alphabet. A ternary kernel with orbital bias is a probability distribution
\[
    \mathbb{P}(\tau = -1) = p_{-}, \quad
    \mathbb{P}(\tau = 0)  = p_{0}, \quad
    \mathbb{P}(\tau = +1) = p_{+},
\]
with $p_{-},p_{0},p_{+} \ge 0$ and $p_{-}+p_{0}+p_{+}=1$. The expected phase increment per bip is
\[
    \mathbb{E}[\Delta k] = \mathbb{E}[\tau] = (-1)p_{-} + 0\cdot p_{0} + (+1)p_{+}.
\]
Given a substrate orbital frequency $\nu_{\mathrm{sub}}$ (in [bip$^{-1}$]) from the calibration in Section~\ref{sec:units} and a ring of length $N$, a target mean phase advance per bip is
\[
    \mu_{\mathrm{orb}} := N\,\nu_{\mathrm{sub}}.
\]
A kernel is said to realise the target orbital frequency if
\[
    \mathbb{E}[\Delta k] = \mu_{\mathrm{orb}} \bmod N.
\]
\end{definition}

\begin{consequence}[Per-bip phase update]\label{cons:phase-update}
On the ring plan with phase index $k_t$ and a ternary kernel as in Definition~\ref{def:orbital-kernel}, a single bip update of the orbital phase is given by:
\begin{enumerate}
    \item draw $\tau_t \in T$ from the declared kernel;
    \item update the phase index by
    \[
        k_{t+1} := (k_t + \tau_t) \bmod N;
    \]
    \item update the Memorum by adding $\tau_t$ as in Definition~\ref{def:phase-memorum}.
\end{enumerate}
The resulting sequence $(k_t)_{t\ge 0}$ is a biased run-and-tumble walk on the ring, with drift controlled by the kernel parameters $(p_{-},p_{0},p_{+})$ and calibrated to the substrate orbital frequency $\nu_{\mathrm{sub}}$.
\end{consequence}

\begin{remark}[Reference implementation]\label{rem:sun-mercury-python}
A reference implementation may realise the update of Consequence~\ref{cons:phase-update} in a script that:
\begin{itemize}
    \item initialises a phase index $k_0$ and ring length $N$;
    \item samples $\tau_t$ from a ternary kernel with declared parameters $(p_{-},p_{0},p_{+})$;
    \item updates $k_t$ according to $k_{t+1} = (k_t + \tau_t) \bmod N$ for $t = 0,\dots,T-1$;
    \item records the phase history $(k_t)$ and, optionally, presents the path in the plane by mapping $k_t \mapsto (\cos(2\pi k_t/N), \sin(2\pi k_t/N))$.
\end{itemize}
The calibration of $(p_{-},p_{0},p_{+})$ and $N$ from external Sun--Mercury data is performed via the units and frequency mapping in Section~\ref{sec:units} and does not modify the AFC update rule.
\end{remark}
